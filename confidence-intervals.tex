\documentclass{article}
\usepackage[english]{babel}
\usepackage[letterpaper,top=2cm,bottom=2cm,left=3cm,right=3cm,marginparwidth=1.75cm]{geometry}
\usepackage{amsmath,amssymb}
\usepackage{accents}
\usepackage{graphicx}
\usepackage[colorlinks=true, allcolors=blue]{hyperref}
\newcommand{\Ifreq}{I_\text{freq}}
\newcommand{\Ibayes}{I_\text{bayes}}
\newcommand{\E}{\mathop{\mathbb{E}}}
\newcommand{\xl}{\underaccent{\bar}{x}}
\newcommand{\xh}{\bar{x}}
\newcommand{\xli}{\xl^{-1}}
\newcommand{\xhi}{\xh^{-1}}
\newcommand{\1}{\mathbf{1}}

\title{The Similarities Between Frequentist Confidence Intervals and Bayesian Credible Intervals}
\author{Ali $\to$ Ben}

\begin{document}
\maketitle

\section{Introduction}

We're a likelihood $p(y|x)$, have observed some data $y$, and wish to represent
our uncertainty in its underlying parameter $x$. We'd like to represent this
uncertainty with an interval function $I(y)$, represented as a lower function
$\xl(y)$ and and upper function $\xh(y)$. A Frequentist might inspect the
likelihood $p(y|x)$ and report a confidence interval function $\Ifreq(y)$ and
guarantee that it covers $x$ 95\% of the time if the data $y$ are re-drawn
infinitely many times.  A Bayesian might instead compute the posterior
$p(x|y)$, and report a "credible" interval function, $\Ibayes(y)$, and
guarantee that it captures 95\% of the probability mass of the posterior.
To simplify matters, we'll consider scalar $x$ and $y$.

In this story, both Frequentists and Bayesians seek an interval function
$I(y)$. But to the Frequentist, the Bayesian makes an untenable assumption:
when $x$ is a universal constant, not a random variable, its posterior is a
nonsensical object.

At the root of this dispute is what each of these camps mean by the word
``probability''.  But I argue that the Frequentist's confidence interval has an
undersirable behavior: In the limit where we seek very low probability
intervals, it does not brack the maximum likelihood estimate. The Bayesian
credible interval, on the other hand, does converge to the maximum a
posterior estimate.


\section{The Bayesian's Credible Interval}

An $\alpha$-credible interval satisfies
\[
\forall_y \Pr_{x|y}\left[ x \in \Ibayes(y) \;|\; y\right] = \alpha.
\]
Since $y$ is random, $\Ibayes(y)$ is random, so the left hand side of this
inequality is also stochastic because it is conditioned on the random variable
$y$. 

The above condition admits many solutions. Any interval
that satisfies the above condition at a particular $y$
can be shifted slightly and re-scaled slightly to obtain
another interval that satisfies the condition.

Among all $\alpha$-credible intervals, we prefer those that are short in some
sense. To do that, we'll cast the above as an optimization problem:
\begin{align}\label{eq:bayes-interval}
\min_{\xl, \xh} &\quad  \| \xh - \xl \|_\infty \\
\text{s.t. } & \min_y \Pr_{x|y}\left[\xl(y) < x < \xh(y)\right] \geq \alpha.
\end{align}
Here, $\|f\|_\infty \equiv \sup_x |f(x)|$.

When $\alpha\to 0$, these intervals converge to the peak of the posterior. In
the next section, I'll show that the penalizing the width of the interval this
way causes the Frequentist interval to not converge to the maximum likelihood
estimate. In preparation for that exposition, let's first see why the Bayesian
interval converges to the MAP estimate.

At opt, the constraint is tight everywhere: $\forall_y
\Pr_{x|y}\left[\xl(y) < x < \xh(y)\right] = \alpha$, and when $\alpha\to 0$,
$\| \xh - \xl \|_\infty \to 0$, meaning that $\forall_y \xh(y) - \xl(y) \to 0$.
A linear approximation of the constraint around opt gives for all $y$,
\begin{align}
        \Pr_{x|y}\left[\xl(y) < x < \xh(y)\right] &= \int_{-\infty}^\infty p(x|y) \1\left[ \xl(y) < x\right] \1\left[x < \xh(y)\right] \;dx\\
        &= \int_{\xl(y)}^{\xh(y)} p(x|y) \;dx\\
        &\approx p_{x|y}(\xl(y) | y) \cdot \left(\xh(y)-\xl(y)\right).
\end{align}
So when $\alpha\to 0$, the optimization problem amounts to
\begin{align}
\min_{\xl, \xh} &\quad  \| \xh - \xl \|_\infty \\
        \text{s.t. } & \forall_y\; p_{x|y}(\xl(y) | y) \cdot \left(\xh(y)-\xl(y)\right) =\alpha,
\end{align}
which can be solved pointwise at each $y$:
\begin{align}
        \min_{\xl(y)\leq \xh(y)} &\quad  \xh(y)-\xl(y) \\
        \text{s.t. } & p_{x|y}(\xl(y) | y) \cdot \left(\xh(y)-\xl(y)\right) =\alpha.
\end{align}
The constraint implies $\xh(y)-\xl(y) = \alpha / p_{x|y}(\xl(y) | y) $. Plugging this in gives the equivalent optimization problem
\[
        \min_{\xl(y)\leq \xh(y)} \quad  \alpha / p_{x|y}(\xl(y) | y),
\]
or more simply, for all $y$,
\[
        \max_{\xl(y)}  p_{x|y}(\xl(y) | y).
\]
meaning that $\xl(y)$ is the peak of the posterior $p(x|y)$, as expected.


\section{The Frequentist's Confidence Interval}

An $\alpha$ confidence interval must satisfy
\begin{equation}
\forall_x\; \Pr_{y|x}\left[x \in \Ifreq(y)\right] = \alpha.
\end{equation}
While $\Ifreq(y)$ is random (its location and width depend on $y$), the left
hand side of the inequality is deterministic. The only source of randomness,
$y$, appears inside the $\Pr$. 

Among all $\alpha$-confidence intervals, we prefer those that are short in some
sense. As before, it's tempting to cast the optimization problem
\begin{align}\label{eq:freq-interval}
\min_{\xl, \xh} &\quad  \| \xh - \xl \|_\infty \\
\text{s.t. } & \min_x \Pr_{y|x}\left[\xl(y) < x < \xh(y) \right] \geq \alpha.
\end{align}
As we take $\alpha\to 0$, $\xl(y) \to \xh(y)$ pointwise as before, but the
interval does not converge to the maximum likelihood $x$. Indeed,
\begin{align}
        \Pr_{y|x}\left[\xl(y) < x < \xh(y)\right] &= \int_{-\infty}^\infty p(y|x) \1\left[ \xl(y) < x\right] \1\left[x < \xh(y)\right] \;dy\\
        &= \int_{-\infty}^\infty p(y|x) \1\left[ y < \xl^{-1}(x)\right] \1\left[\xl^{-1}(x) < y\right] \;dy\\
        &= \int_{\xh^{-1}(x)}^{\xl^{-1}(x)} p(y|x)  \;dy\\
        &\approx p_{y|x}\left(\xh^{-1}(x) | x\right) \cdot \left(\xl^{-1}(x)-\xh^{-1}(x)\right).
\end{align}
Replacing the constraint with this approximation near opt gives, as $\alpha\to 0$,
\begin{align}\label{eq:non-mle}
  \min_{\xl, \xh} &\quad  \| \xh - \xl \|_\infty \\
  \text{s.t. } & \forall_x\; p_{y|x}\left(\xh^{-1}(x) | x\right) \cdot \left(\xl^{-1}(x)-\xh^{-1}(x)\right) = \alpha.
\end{align}
To simplify this, we'll use $\frac{\xh(y) - \xl(y)}{\xl^{-1}(x)-\xh^{-1}(x)} \approx \dot{\xh}(y)$ when $x = \xh(y)$ to write the problem as
\begin{align}
\min_{\xl^{-1}, \xh^{-1}} &\quad  \| \left(\xl^{-1} - \xh^{-1} \right) \cdot \dot{\xl}^{-1}\|_\infty \\
\text{s.t. } & \forall_x\; p_{y|x}\left(\xh^{-1}(x) | x\right) \cdot \left(\xl^{-1}(x)-\xh^{-1}(x)\right) = \alpha,
\end{align}

The presence of $\dot{\xl}^{-1}$ in the objective means we can't solve this
problem pointwise. In particular, our choice of $\ell_\infty$ norm to penalize
the width of the interval implies that the interval does not converge to the
maximum likelihood estimate. The one exception to this is when $\xli$ is
linear, so that its derivative is constant. This happens, for example, when $x$
is the location parameter of a distribution.

\section{A Frequentist Confidence Interval That Converges to the Maximum Likelihood Estimate}

Here is a Frequentist confidence interval that does converge to the maximum
likelihood estimate of $x$ as $\alpha\to 0$. Instead of penalizing
$\|\xh-\xl\|_\infty$, we'll penalize the width of the inverse of the confidence
interval:
\begin{align}
\min_{\xl, \xh} &\quad  \| \xl^{-1} - \xh^{-1} \|_\infty \\
\text{s.t. } & \min_x \Pr_{y|x}\left[\xl(y) < x < \xh(y) \right] \geq \alpha.
\end{align}
As we take $\alpha\to 0$, $\xh^{-1}(x) \to \xl^{-1}(x)$ pointwise, and equation (\ref{eq:non-mle}) becomes
\begin{align}
  \min_{\xl, \xh} &\quad  \| \xl^{-1} - \xh^{-1} \|_\infty \\
  \text{s.t. } & \forall_x\; p_{y|x}\left(\xh^{-1}(x) | x\right) \cdot \left(\xl^{-1}(x)-\xh^{-1}(x)\right) = \alpha.
\end{align}
We can solve this problem pointwise for each $x$:
\begin{align}
  \min_{\xli(x)> \xhi(x)} &\quad  \xli(x) - \xhi(x) \\
  \text{s.t. } &  p_{y|x}\left(\xhi(x) | x\right) \cdot \left(\xli(x)-\xhi(x)\right) = \alpha.
\end{align}
After substituting the constraint into the objective, this gives
\begin{align}
  \max_{\xhi(x)} &\quad  p_{y|x}\left(\xhi(x) | x\right) 
\end{align}
In other words, for each x, $\xhi(x)$ the most probable observation under
$x$, or equivalently, for each observation $y$, $\xh(y)$ is the $x$ under which
$y$ is most probable. That's exactly the definition of the maximum likelihood
estimate of $x$ under an observation $y$.
 
\section{Conclusion}

I've shown that the obvious ways to penalize the width of a confidence interval
gives confidence intervals that don't converge to the MLE. Neyman's 1933 paper
on the subject offers a variety of other ways to canonicalize intervals, but
some of them don't admit solutions, and as far as I can see, none of them
converge to the MLE. For the confidence to converge to the MLE requires an
uncomfortable penalty on its width.

\end{document}
